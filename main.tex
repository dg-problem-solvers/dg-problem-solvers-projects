\documentclass{amsart}

%\documentclass[10 pt]{amsart}

\usepackage[ocgcolorlinks,linktoc=all]{hyperref}
\hypersetup{citecolor=blue,linkcolor=red}

%\usepackage{amsthm}
\usepackage{cleveref}
\crefname{lemma}{Lemma}{Lemmata}
\crefname{equation}{equation}{equations}

\newtheorem{theorem}{Theorem}
\newtheorem{lemma}[theorem]{Lemma}
\newtheorem{proposition}[theorem]{Proposition}
\newtheorem{corollary}[theorem]{Corollary}

\newtheorem*{thmA}{Theorem}
\newtheorem*{thmB}{Theorem}
\newtheorem*{rem}{Remark}
\newtheorem*{thmmain}{Theorem}
\newtheorem*{propmain}{Proposition}

\theoremstyle{definition}
\newtheorem{definition}[theorem]{Definition}
\newtheorem{example}[theorem]{Example}
\newtheorem{xca}[theorem]{Exercise}

\theoremstyle{remark}
\newtheorem{remark}[theorem]{Remark}

\numberwithin{equation}{section}

%Symbols
\renewcommand{\~}{\tilde}
\renewcommand{\-}{\bar}
\newcommand{\bs}{\backslash}
\newcommand{\cn}{\colon}
\newcommand{\sub}{\subset}

\newcommand{\N}{\mathbb{N}}
\newcommand{\R}{\mathbb{R}}
\newcommand{\Z}{\mathbb{Z}}
\renewcommand{\S}{\mathbb{S}}
\renewcommand{\H}{\mathbb{H}}
\newcommand{\C}{\mathbb{C}}
\newcommand{\K}{\mathbb{K}}
\newcommand{\Di}{\mathbb{D}}
\newcommand{\B}{\mathbb{B}}
\newcommand{\8}{\infty}

%Greek letters
\renewcommand{\a}{\alpha}
\renewcommand{\b}{\beta}
\newcommand{\g}{\gamma}
\renewcommand{\d}{\delta}
\newcommand{\e}{\epsilon}
\renewcommand{\k}{\kappa}
\renewcommand{\l}{\lambda}
\renewcommand{\o}{\omega}
\renewcommand{\t}{\theta}
\newcommand{\s}{\sigma}
\newcommand{\p}{\varphi}
\newcommand{\z}{\zeta}
\newcommand{\vt}{\vartheta}
\renewcommand{\O}{\Omega}
\newcommand{\D}{\Delta}
\newcommand{\G}{\Gamma}
\newcommand{\T}{\Theta}
\renewcommand{\L}{\Lambda}

%Mathematical operators
\newcommand{\INT}{\int_{\O}}
\newcommand{\DINT}{\int_{\d\O}}
\newcommand{\Int}{\int_{-\infty}^{\infty}}
\newcommand{\del}{\partial}
\DeclareMathOperator{\grad}{\nabla}

\newcommand{\inpr}[2]{\left\langle #1,#2 \right\rangle}
\newcommand{\fr}[2]{\frac{#1}{#2}}
\newcommand{\x}{\times}

\DeclareMathOperator{\dive}{div}
\DeclareMathOperator{\id}{id}
\DeclareMathOperator{\pr}{pr}
\DeclareMathOperator{\Diff}{Diff}
\DeclareMathOperator{\supp}{supp}
\DeclareMathOperator{\graph}{graph}
\DeclareMathOperator{\osc}{osc}
\DeclareMathOperator{\const}{const}
\DeclareMathOperator{\dist}{dist}
\DeclareMathOperator{\loc}{loc}

%Environments
\newcommand{\Theo}[3]{\begin{#1}\label{#2} #3 \end{#1}}
\newcommand{\pf}[1]{\begin{proof} #1 \end{proof}}
\newcommand{\eq}[1]{\begin{equation}\begin{alignedat}{2} #1 \end{alignedat}\end{equation}}
\newcommand{\IntEq}[4]{#1&#2#3	 &\quad &\text{in}~#4,}
\newcommand{\BEq}[4]{#1&#2#3	 &\quad &\text{on}~#4}
\newcommand{\br}[1]{\left(#1\right)}

%Diff Geo commands
\DeclareMathOperator{\ric}{\text{Ric}}

%Logical symbols
\newcommand{\Ra}{\Rightarrow}
\newcommand{\ra}{\rightarrow}
\newcommand{\hra}{\hookrightarrow}
\newcommand{\mt}{\mapsto}

%Fonts
\newcommand{\mc}{\mathcal}
\renewcommand{\it}{\textit}
\newcommand{\mrm}{\mathrm}

%Spacing
\newcommand{\hp}{\hphantom}


\parindent 0 pt

\protected\def\ignorethis#1\endignorethis{}
\let\endignorethis\relax
\def\TOCstop{\addtocontents{toc}{\ignorethis}}
\def\TOCstart{\addtocontents{toc}{\endignorethis}}


\begin{document}

\title{Projects}

\curraddr{}
\email{}
\date{\today}

\dedicatory{}
\subjclass[2010]{}
\keywords{}

\begin{abstract}
\end{abstract}

\maketitle

\section{Introduction}

A list of projects and some notes on those projects.

\section{Gauss Parametrisation On The Sphere}

The aim here is to obtain a parametrisation of a family of convex bodies in the sphere, analogous to the Gauss-map parametrisation in Euclidean space. 

\begin{itemize}
\item Via projections. 

The idea is to define a projection map $\mathbb{S}^{n+1} \to \mathbb{R}^{n+1}$. This should at the very least take convex bodies to convex bodies. These projections generally are ``based at a point'', $p_0$. Ideally these should also be conformal. Some candidates (not all conformal) are
\begin{itemize}
\item Gnomic Projection,
\item Steriographic Projection,
\item Vertical projection (Orthogonal projection onto the hyperplane orthogonal to $p_0$).
\end{itemize}
Now we have a Gauss-map on $\mathbb{R}^{n+1}$ using only the differentiable structure on $\mathbb{R}^{n+1}$ and so our Gauss map is composition of the Euclidean Gauss map with the projection map. Equivalently, we can just work on $\mathbb{R}^{n+1}$ with the metric induced by the projection and completely forget about the sphere. The best candidate seems to be steriographic projection since it realises there sphere as locally conformally flat (all we miss is a single point!) and the conformal factor is rotationally symmetric.

For this to work, the base point needs to be chosen correctly. For shrinking flows that collapse to a point, this should be the point of collapse, and for expanding flows that converge to an equator, this should be either centre of the equator. One needs to take a little care with the class of solutions that exist for all time and don't collapse, e.g. closed curves bounding an area equal to $2\pi$ will exist for all time and converge to an equator under the curve shortening flow (a shrinking flow), so really should be treated like expanding flows.

\item Support function parametrisations.

Here we need a translation map $\tau$ taking the normal $\nu$ to a fixed ``base point'' $p_0$ and above. Then the support function should measure the distance from the base point to the tangent equator. As a function of $x\in M$ the hypersurface, this should be $s(x) = \nu(x) \cdot d \grad d$ where $d$ is the distance from $p_0$ to $x$. In geodesic polar coordinates, $d \grad d = r \partial_r$. Our translation map has to have some uniqueness properties, in particular, a convex hypersurface should provide a diffeomorphism of the hypersurface onto the unit sphere $U_{p_0} \mathbb{S}^{n+1} \subset T_{p_0} \mathbb{S}^{n+1}$. Then to parametrise the hypersurface we first do
\[
z \in U_{p_0} \mathbb{S}^{n+1} \mapsto y = \exp_{p_0} (s(z) z)
\]
The equator $E_x$ through $y$ perpendicular to the geodesic connecting $p_0$ to $y$ (i.e. $\gamma(t) = \exp_{p_0} (t z)$ with $\gamma(s(z)) = y$) should intersect $M$ in precisely one point $x$, which is the point with $\tau \nu = z$. Letting $\alpha(z) = d(y, x)$ and \(v(z) \in T_y E\) such that \(x = \exp_y(\alpha(z) v(z))\), our parametrisation now becomes
\[
\varphi: z \in U_{p_0} \mathbb{S}^{n+1} \mapsto \exp_{\exp_{p_0} (s(z) z)} (\alpha(z) v) = \exp_{y} (\alpha(z) v(z)).
\]

If this is analogous to the Euclidean space situation, \(\alpha(z) v(z) = \tau_{p_0,y} \grad s (z)\) where \(\tau_{p_0,y}\) denotes parallel translation along the geodesic \(\gamma(t) = \exp_{p_0} (t z)\) from the point \(p_0\) (\(t=0\)) to the point \(y\) (\(\gamma(s(z))\). 
\end{itemize}

All this being resolved, then changing from one of the parametrisations above to the flow parametrisation should take the Harnack inequality $\partial_t F \geq 0$ to the Harnack inequality \(\partial_t F + \|\mathcal{W}\|_{\dot{F}} \geq 0\). Perhaps it takes it to something else and this ``something else'' is in fact the sharp Harnack inequality. All of this seems difficult to compute! It may be easiest to work in the conformally flat model offered by steriographic projection where calculations could be rather more explicit.

\section{Type II Singularity Classification}

The aim here is to classify Type II singularities of Mean Convex MCF (or more general flows, why not?) as Bowl solitons. A possible approach is to obtain a Harnack inequality for the non-collapsing two-point function $k$ and show equality is attained precisely on rotationally symmetric translating solitons, hence the Bowl soliton by uniqueness. In fact, we may only need a Harnack inequality for translating graphs (are all translators graphs over the hyperplane orthogonal to the translation direction?) in which equality is attained precisely on rotationally symmetric translating graphs!

\section{Andrews Gauss Curvature Flow Conjecture}

The aim is to show that the Gauss curvature flow of convex, closed hypersurfaces converges to round points in all dimensions. Some possible approaches are as follows:

\begin{itemize}
\item Isoperimetric profile comparisons. 

Take a pair of geometric functionals, such as enclosed volume and area, or a pair of mixed volumes, and define the profile $I(x)$to be the least of one functional subject to the constraint that the other functional is $x$. A variation on this theme is Hamilton's isoperimetric ratio which deals with dividing a set into two pieces and minimising some combination of these (Hamilton uses Harmonic mean) subject to a constraint on the common boundary. 

The idea is to find a comparison function that is initially below the initial profile of a hypersurface, and satisfies a differential inequality ensuring it remains below. If it converges to the sphere profile, then we are done because it is already known that a limit exists. In general we don't expect the profile to be monotone, but it should have some concavity properties (from the convexity of the hypersurface), as should the comparison which might be sufficient to obtain the result. 

It might be better to study the isoperimetric profile of the domain bounded by the hypersurface instead. Since this is a convex body, the result in \cite{MR1674097} says the isoperimetric profile is concave. If you take the profile of the hypersurface itself, I don't know whether it's concave or not. It would be awesome to be able to construct a comparison with some model solution, but this may not actually be possible. See \cite{MR2843240} for the corresponding result for the CSF.

\item Comparison Geometry I: Riccati Equation.

Consider the comparison geometry operator $A$, geodesics $\gamma$ as the tangent space endomorphism $A = \nabla \gamma'$. This is trivial in the direction $\gamma'$ since $\gamma$ is a geodesic. For constant sectional curvature $K_0$ spaces, $A = K_0 Id$ on normal directions. Perhaps it will be possible to show that $A$ converges to $K_0 Id$ which then gives the result since constant sectional curvature spaces are characterised by this identity. 

The operator $A$ satisfies the Riccati equation
\[
A' + A^2 + \ric(\gamma') = 0.
\]
Jacobi fields are solutions of
\[
J' = A J
\]
This is going to be closely related to Perelman's $\mathcal{L}$-geodesics theory.

\item Comparison Geometry II: Heintze-Karcher evolution.

The Heintze-Karcher inequality arises from integrating along normal geodesic rays until you hit the cut locus of the hypersurface. A centrally-symmetric body (in particular the sphere!) has cut locus a single point (I think). Now the equality case in Hientze-Karcher is again attained on the sphere. So we can see how some quantity cooked up from Heintze-Karcher evolves and show it converges to the sphere equality case. 

Variations on this would be to consider following normal geodesic rays until you hit the cut-locus and define a scalar valued function that assigns the distance from this point to the cut-locus. If this converges to a constant, surely we must have a sphere!

\item Symmetrisation techniques

I wonder what happens to some sort of symmetrisation map acting on the hypersurface under the flow? E.g. if one applies a Steiner (Schwarz, some sort of Aleksandrov reflection?) symmetrisation at each time, does this somehow improve under the flow? More precisely, choose a hyperplane through the origin and Steiner symmetrise with respect to this plane. Now try to show that the Hausdorff distance from the hypersurface to it's symmetrisation is monotone decreasing under the flow. Or perhaps there is some sort of comparison with a model that converges to $0$. Maybe Hausdorff distance is not the best choice. Is there some other way to measure how far a hypersurface deviates from it's symmetrisation? Another intriguing idea would be to think of this as a function on the sphere. Given a point on the sphere (a direction), symmetrise with respect to the orthogonal hyperplane and then measure (e.g. Hausdorff distance) how far the symmetrisation is from the original hypersurface. This produces a one parameter family of real valued functions defined on the sphere. Show this converges to $0$ under the flow and you're done!
\end{itemize}

\section{Classification of (Quasi) Ancient Solutions on the Sphere}

\begin{itemize}
\item Non-collapsing leads to very short argument for classification with 1-homogeneous speeds.
\item Convex Rigidity + Aleksandrov reflection gives a classification for any imaginable allowable speed, even in the quasi-ancient case.
\item What are the equivalent conditions to \cite{2014arXiv1405.7509H} for convex ancient solutions on the sphere?
\end{itemize}

\section{Harnack and Ancient solutions on Riemannian Manifolds with Positive Curvature}

\begin{itemize}
\item Possibly two projects?
\item If $\ric \geq K_0 g$, for $K_0 \geq 0$, do we obtain a Harnack inequality?
\item Use centre manifold analysis to show ancient solutions exist near minimal hyper-surfaces and to obtain uniqueness. Convex ancient solutions should correspond to totally geodesic hyper-surfaces.
\end{itemize}

\section{(Quasi) Ancient solutions with singular backwards limits}

\begin{itemize}
\item For quasi-ancient solutions of the flow $H^p$, $0<p<1$ in the sphere, if $H$ is bounded then Rigidity + Alexsandrov reflection gives classification.
\item Are there quasi-ancient solutions with $H$ unbounded backwards in time?
\item Given any convex polyhedra, we should be able to use the Harnack inequality (like Hamilton, ref?) to show that there is a unique solution emanating from the polyhedron. The question is, can such solutions be quasi-ancient? If the polyhedron is strictly convex, the answer is no, because we have a barrier starting with a geodesic sphere that is not an equator. But if the polyhedron intersects an equator in at least 2 points, no such initial barrier exists so we cannot immediately rule these out. The example to bear in mind here is that of Lunes. Can we estimate the speed at which the singular backwards limit pulls away from the equator to then insert a barrier at time $-T_S + \epsilon$ for which the flow exists for strictly less time than $-T_S + \epsilon$? This would rule out non-singular limits. On the other hand, could we perhaps show that singular limits do exists by constructing one?
\item Blow up analysis: Set $\lambda_k := \sup\limits_{(x,t)\in M\times [-t_k,-\varepsilon]} H(x,t),$ where $t_k$ approaches $-T_S$ and $\varepsilon>0$ is small enough. Set $\lambda_{\infty}=\lim_{k\to\infty}\lambda_k$ and $x_k(\cdot,t):=(\lambda_k+1)x(\cdot, t_k+(\lambda_k+1)^{-p-1}t)$. Then if $\lambda_{\infty} < \infty$ it should be possible to show $x_{\infty}$ is an equator. If $\lambda_{\infty} = \infty$, then the $x_{\infty}$ is a translating soliton. In the latter case, there is a singular backwards limit! The question is whether such things can exist. Do Lunes, or more generally (non-strictly) convex polyhedra blow up to look like translating solitons?
\end{itemize}

\section{$L^p$ Minkowski Problem}

\section{Forward Convergence Results}

\section{Optimality of Harnack and Solitons}

\section{Alexandrov-Fenchel inequalities}

\section{Convergence to spherical floating bodies}



\bibliographystyle{amsplain}
\bibliography{Bibliography.bib}


\end{document}
